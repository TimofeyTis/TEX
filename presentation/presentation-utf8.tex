\documentclass[pdf, 12pt, aspectratio=169, bigger, unicode]{beamer}


\usepackage[utf8]{inputenc}
\usepackage[T1, T2A]{fontenc}
\usepackage[english,russian]{babel}
\usepackage{pscyr}

\usepackage{graphics}
\usepackage{graphicx}
\usepackage{graphpap}
\usepackage{ragged2e}
\usepackage{amsfonts}
\usepackage{amsmath}
\usepackage{amssymb}
\usepackage{amstext}
\usepackage{indentfirst}
\usepackage{amsthm}
\usepackage{hyperref}
\usepackage{tikz}
\usepackage{pgf}
\usepackage{pgfplots}
\usepackage{multicol}
\usepackage{slashbox}
\usepackage{tabularx}
\usepackage{bibunits}
\usepackage{epstopdf}
\usepackage{algorithmicx}

 
\usepackage{float}
\usepackage[ruled]{algorithm}
\usepackage{algpseudocode}



\usefonttheme[onlymath]{serif}
\usetikzlibrary{decorations}
%Нужно включать, если используется "тема" (стиль оформления) по умолчанию
\usepackage{beamerthemesplit}
\usetheme{Warsaw}


\usecolortheme{seahorse}


%\captionsetup[ruled]{labelsep=period}

\renewcommand{\thealgorithm}{\arabic{algorithm}}
\floatname{algorithm}{Алгоритм}
\algrenewcommand\algorithmicrequire{\textbf{Вход: }}
\algrenewcommand\algorithmicensure{\textbf{Выход: }}
\algrenewcommand\algorithmicwhile{\textbf{До тех пока}}
	\algrenewcommand\algorithmicdo{\textbf{выполнять}}
	\algrenewcommand\algorithmicrepeat{\textbf{Повторять}}
	\algrenewcommand\algorithmicuntil{\textbf{Пока выполняется}}
	\algrenewcommand\algorithmicend{\textbf{Конец}}
	\algrenewcommand\algorithmicif{\textbf{Если}}
	\algrenewcommand\algorithmicelse{\textbf{иначе}}
	\algrenewcommand\algorithmicthen{\textbf{тогда}}
	\algrenewcommand\algorithmicfor{\textbf{Для}}
	\algrenewcommand\algorithmicforall{\textbf{Выполнить для всех}}
	\algrenewcommand\algorithmicfunction{\textbf{Функция}}
	\algrenewcommand\algorithmicprocedure{\textbf{Процедура}}
	\algrenewcommand\algorithmicloop{\textbf{Зациклить}}
	\algrenewcommand\algorithmicreturn{\textbf{Возвратить}}
	\algrenewtext{EndWhile}{\textbf{Конец цикла}}
	\algrenewtext{EndLoop}{\textbf{Конец зацикливания}}
	\algrenewtext{EndFor}{\textbf{Конец цикла}}
	\algrenewtext{EndFunction}{\textbf{Конец функции}}
	\algrenewtext{EndProcedure}{\textbf{Конец процедуры}}
	\algrenewtext{EndIf}{\textbf{Конец условия}}
	\algrenewtext{EndFor}{\textbf{Конец цикла}}
	\algrenewtext{BeginAlgorithm}{\textbf{Начало алгоритма}}
	\algrenewtext{EndAlgorithm}{\textbf{Конец алгоритма}}
	\algrenewtext{BeginBlock}{\textbf{Начало блока. }}
	\algrenewtext{EndBlock}{\textbf{Конец блока}}
	\algrenewtext{ElsIf}{\textbf{иначе если }}
		
\algloop{description}
	\algnewcommand\algorithmicdescription{\textbf{Описание алгоритма}}


\mode<presentation>
{
    \usefonttheme[onlymath]{serif}
		%\usetheme{Copenhagen}
        %\usetheme{Warsaw}
        %\usetheme{Darmstadt}
        %\usetheme{Frankfurt}
        %\usetheme{AnnArbor}
        %\usetheme{CambridgeUS}
    \setbeamercovered{transparent}	
	\setbeamertemplate{footline}
{%
 \leavevmode%
    \hbox{%
        \begin{beamercolorbox}[wd=.5\paperwidth,ht=2.5ex,dp=1.125ex,leftskip=.3cm,rightskip=.3cm]{author in head/foot}%
            \usebeamerfont{author in head/foot}%\insertframenumber{}/\inserttotalframenumber%
                        \hfill\insertshortauthor
        \end{beamercolorbox}%
        \begin{beamercolorbox}[wd=.5\paperwidth,ht=2.5ex,dp=1.125ex,leftskip=.3cm,rightskip=.3cm plus1fil]{title in head/foot}%
            \usebeamerfont{title in head/foot}\insertshorttitle
						\hfill \insertframenumber{}/\inserttotalframenumber%
        \end{beamercolorbox}}%
    \vskip0pt%
}
\setbeamertemplate{navigation symbols}{} 
\setbeamertemplate{navigation symbols}{}
}

\bibliographystyle{unsrt}
\makeatletter
\addto\captionsrussian{
    %\renewcommand{\figurename}{Рисунок}
    \def\figurename{Рисунок}
}


%Более крупный шрифт для подзаголовков титульного листа
\setbeamerfont{institute}{size=\normalsize}

%Задание команды (\bluetext) для выделения конкретным (синим) цветом
%(используйте \alert для выделения цветом выбранной "темы")
\setbeamercolor{bluetext_color}{fg=blue}
\newcommand{\bluetext}[1]{{\usebeamercolor[fg]{bluetext_color}#1}}

\renewcommand{\raggedright}{\leftskip=0pt \rightskip=0pt plus 0cm}

\renewcommand{\rmdefault}{ftm}

%Если используется последовательное появление пунктов списков на слайде
%(не злоупотребляйте в слайдах для защиты дипломной работы), чтобы
%еще непоявившиеся пункты были все-таки немножко видны.
\setbeamercovered{transparent}

\title[ЗАДАЧА MARL ДЛЯ СВЕТОФОРА НА ПЕРЕКРЁСТКЕ]{ЗАДАЧА MARL ДЛЯ СВЕТОФОРА НА ПЕРЕКРЁСТКЕ}
\author[Тисленко Т.И.]{ \bf Тисленко Тимофей Иванович}
\institute[ИМиФИ СФУ]{	{\footnotesize ФГАОУ ВО <<СИБИРСКИЙ ФЕДЕРАЛЬНЫЙ УНИВЕРСИТЕТ>>\\

    Институт математики  и фундаментальной информатики\\[-2pt]
		
	%Кафедра высшей и прикладной математики
	}

 % \vspace{0.5cm}
%{\footnotesize Направление \ 01.037.02 Прикладная математика }

  \vspace{0.2cm}
    {\footnotesize Научный руководитель --- к.ф.-м.н.,  доцент   {\sf Д.В. Семенова }					}
 }
\date[\today]{\footnotesize Томск, МПОИТЭС	 2021}

\begin{document}

%%%%%%%%%%%%%%%%%%%%%%%%%%%%%%%%%%%%%%%%
\begin{frame}
\titlepage

 \end{frame}

\normalsize
%%%%%%%%%%%%%%%%%%%%%%%%%%%%%%%%%%

\begin{frame}
	\frametitle{Актуальность}
    \justifying
    В Красноярске стремительно с каждым годом растет кол-во желающих стать автомобилистом.
    Так, Красноярский край оказался на 12 месте в топ-20 регионов по объему автомобильного парка в России по данным агентства «Автостат» на 1 января 2020.
    С количеством автомобилистов раздувается и время, которое проводится в пробках.
\end{frame}
%%%%%%%%%%%%%%%%%%%%%%%%%%%%%%%%%%

\begin{frame}
    \frametitle{Существующие модели адаптивыных систем светофоров}
    \justifying
    \begin{figure}[h!]
		\centering
            
			\includegraphics[scale = 0.8]{images/issues.png}
		    \caption{ }
            \label{fig:issues}
	\end{figure}


\end{frame}
%%%%%%%%%%%%%%%%%%%%%%%%%%%%%%%%%%

\begin{frame}
    \frametitle{Цели и задачи}
    \justifying
    \begin{block}{Цель работы}
        \justifying
        Разработка и исследование математической модели мультиагентной системы для задачи оптимизации движения на перекреске.
    \end{block}

    %\pause

    \begin{block}{Задачи}
        \begin{enumerate} \justifying
            \item Сделать обзор литературы по соответствующей тематике.
            \item Описать математическую модель.
            \item Описать алгоритм.
            \item Продемонстрировать результаты работы алгоритма.
        \end{enumerate}
    \end{block}

\end{frame}

%%%%%%%%%%%%%%%%%%%%%%%%%%%%%%%%%%%%%%%%%%%%%%%%%%%%%%%%%
\begin{frame}
    \justifying
    \frametitle{Основные определения}

    \begin{block}{Определения:} 
        \begin{enumerate} \justifying
            \item	\textbf{Интеллектуальным агентом} называется метаобъект, наделенный долей субъектности,
            взаимодействующий с другими агентами и средой, выполняющий определенные функции для достижения поставленных целей.
            \item	\textbf{Средой} называется множество объектов, не принадлежащих агенту.
            \item 	\textbf{Задачами/ресурсами} называются объекты, распределяемые агентами в ходе достижения целей.
            \item 	\textbf{Мультиагентная система} – совокупность взаимосвязанных агентов.
            \item 	\textbf{RL(Reinforcement Learning)} — Обучение с подкреплением, где
            в роли учителя выступает среда. 
        \end{enumerate}
    \end{block}
					


\end{frame}

%%%%%%%%%%%%%%%%%%%%%%%%%%%%%%%%%%%%%%%%%%%%%%%%%%%%%%%%%
\begin{frame}
\justifying
\frametitle{Постановка задачи}

	
	\begin{figure}[h!]
		\centering
            
			\includegraphics[scale = 1.8]{images/intersection2.png}
            \label{fig:intersection1}
		    \caption{ }
	\end{figure}

  

\end{frame}

%%%%%%%%%%%%%%%%%%%%%%%%%%%%%%%%%%%%%%%%%%%%%%%%%%%%%%%%%
\begin{frame}
\justifying
\frametitle{}
\begin{block}\justifying
    \begin{enumerate}[-] \justifying    
        \item Модель среды – Марковский процесс принятия решения-
        \item агент --- светофор
        \item cреда --- перекресток, на котором на отрезках дорог за 100м до стоп-линий засекается время.
        \item пространство состояний $S = \{$ $s_0$ = <<фаза0>> , $s_1$ = <<фаза1>>  $\}$
        \item в момент времени $t_k$ активна фаза светофора $S_k$, суммарное засеченое всех машин, 
        проходящих через отрезок дороги называется задержкой на фазе $S_k$ 
        \item множество действий A = $\{$ $a_0$ = <<оставить фазу>> , $a_1$ = <<сменить фазу>> $\}$
    \end{enumerate}
\end{block}


\end{frame}

%%%%%%%%%%%%%%%%%%%%%%%%%%%%%%%%%%%%%%%%%%%%%%%%%%%%%%%%%
\begin{frame}
\justifying
    \begin{figure}[h!]
        \centering
            
            \includegraphics[scale = 0.6]{images/state-action.png}
            \label{fig:scheme}
            \caption{ }
    \end{figure}

	\begin{block}\justifying
        \begin{enumerate}[-] \justifying  
        \item $r(s, a)$ = задержка на фазе $s$ после действия $a$
        \item $p(i, k; j)$ вероятность того, что система из состояния $i$ при выборе решения $k$ попадает в состояние $j$,
        полностью определяется состоянием, в которое переходит процесс.
        \end{enumerate}
    \end{block}


\end{frame}

%%%%%%%%%%%%%%%%%%%%%%%%%%%%%%%%%%%%%%%%%%%%%%%%%%%%%%%%%
\begin{frame}
\justifying

	$V^*(s)$ --- функция суммарных внешних доходов от оптимальной политики в состоянии s
	\begin{equation}\label{VAL}
		V^*(s)  = 
		\max_{a(\cdot )} \sum_{t =0 }^{\infty} \gamma ^t r(s_t, a_t).
	\end{equation}



	Уравнение Вальда - Беллмана ~\cite{LEC} для управляемого марковского процесса с конечным числом действий и состояний имеет вид:
	\begin{equation}\label{BELLMAN}
		V^*(s)  = 
		\max_{a\in A} \left\{ \sum_{s'\in S} p(s, a; s') (r(s, a; s') +  \gamma V^* (s') )\right\}.
	\end{equation}

	Т.е.  
    \begin{multline*}
        V^*(S_0) = 
        max \{
            p( S_0, A_0;S_1)(r( S_0, A_0) + \gamma V^*(S_1) 
        + p( S_0, A_0;S_1)(r( S_0, A_0) + \gamma V^*(S_0) ; \\
        p( S_0, A_1;S_1)(r( S_0, A_1) + \gamma V^*(S_1) 
        + p( S_0, A_1;S_1)(r( S_0, A_1) + \gamma V^*(S_0) 
        \}
    \end{multline*}


\end{frame}

%%%%%%%%%%%%%%%%%%%%%%%%%%%%%%%%%%%%%%%%%%%%%%%%%%%%%%%%%
\begin{frame}
\justifying
    \begin{equation}
        V^*(s) = \max_{a\in A}  Q(s, a) = \max_{a\in A}  Q_t(s, a).
    \end{equation}

	\begin{equation}
		Q(s, a) = \sum_{s'\in S} p(s, a; s')(r(s, a; s')) +  \gamma V^* (s').
	\end{equation}
	
	Идея Q-обучения заключается в оценке невычислимой правой части:
	\begin{equation} \label{Qiteration}
        Q_{t+1}(s, a) =  Q_t(s, a) + \alpha _t (s, a) \left(r(s, a) +  \gamma \max_{a'\in A}  Q_t(s', a') - Q_t(s, a)\right)
    \end{equation}
    где $s'$--– положение процесса на шаге $t+1$, если на шаге $t$ процесс был в состоянии $s$ и было выбрано действие $a$. Если на шаге $t$ процесс находился в состоянии $s$ и было выбрано действие a, то $0 \leq \alpha _t(s, a) \leq 1$, иначе $\alpha _t(s, a) = 0$.
\end{frame}

%%%%%%%%%%%%%%%%%%%%%%%%%%%%%%%%%%
\begin{frame}
\frametitle{Сходимость}
\justifying
	$Q = \{ Q(s, a)\}_{s \in S, a' \in A}$ , можно записать итеративно $Q_{t+1} = A(Q_t)$, где $A \colon \mathbb{R} _{\infty}^1 \to  \mathbb{R} _{\infty}^1$ --- сжимающее отображение.

    \begin{multline*}
    \rho ((A \circ Q_1)(s, a), (A \circ Q_2)(s, a)) % = \\
     %   = \max_{a'\in A, s'\in S}| \sum_{s'\in S}  p(s, a; s')(r(s, a) + \gamma \max_{a'\in A}  Q_1(s', a')) - \\
   %  - \sum_{s'\in S}  p(s, a; s')(r(s, a) + \gamma \max_{a'\in A}  Q_2(s', a'))|  
    \leq  \max_{a'\in A, s'\in S} |\gamma \max_{a'\in A}  Q_1(s', a')) - \gamma \max_{a'\in A}  Q_2(s', a'))| = \\
        = 	\gamma \rho (Q_1(s, a), Q_2(s, a)) ,  \gamma \in (0; 1)
    \end{multline*}

\end{frame}

%%%%%%%%%%%%%%%%%%%%%%%%%%%%%%%%%%
\begin{frame}
\frametitle{}
\justifying
	Оказывается, что если используемая стратегия $a(\cdot )$ приводит к тому, что с вероятностью 1 каждая пара $(s, a)$ будет бесконечное число раз встречаться на бесконечном горизонте наблюдения, то из отмеченного выше условия сжимаемости при
    \begin{equation}
        \sum _{t=0} ^\infty {\alpha _t (s, a)} = \infty, 
        \sum _{t=0} ^\infty {\alpha _t (s, a)^2} \leq \infty
    \end{equation}
    удет следовать сходимость (с вероятностью 1)  процесса \ref{Qiteration}

    \begin{equation}
	    V ^*(s) = \max_{a\in A}  \lim _{t \to + \infty}Q_t(s, a).
    \end{equation}
    \begin{equation}
	    a_t(s) = \arg  \max_{a' \in A} Q_t(s, a')
    \end{equation}	

\end{frame}

%%%%%%%%%%%%%%%%%%%%%%%%%%%%%%%%%%
\begin{frame}
\frametitle{Основные результаты работы}
\justifying
Целью работы являлось ознакомление с подходами, позволяющими оптимизировать процесс выбора сигнала светофора, с учетом текущей загрузки транспорта, с точки зрения минимизации задержки. В работе получены следующие результаты:

\begin{enumerate}
	\item Математическая модель процесса выбора фазы светофора, отличающаяся учетом текущего расположения светофоров и их загрузки и позволяющая сформулировать оптимизационные задачи, целью которых является минимизация задержки трафика автомобилей.
	\item Структура мультиагентной системы, включающая в себя единственного агента – светофор, обеспечивающая наиболее эффективное распараллеливание всей задачи на подзадачи, которые будут решены агентом.
\end{enumerate}	


\end{frame}

%%%%%%%%%%%%%%%%%%%%%%%%%%%%%%%%%%
\begin{frame}
\begin{thebibliography}{2}
	\bibitem{LEC}
		Лекции по случайным процессам : учебное пособие /
		А. В. Гасников, Э. А. Горбунов, С. А. Гуз и др. ; под ред.
		А. В. Гасникова. – <<Москва>> : МФТИ, 2019. – 285 с.
		ISBN 978-5-7417-0710-4
	\bibitem{BOOK}
		Марковские процессы принятия решений. /
		Майн X., Осаки С. 
		Главная редакция физико-математической литературы издательства «Наука», 
		1977.  - 176 с. 
		УДК 519.283
	%
\end{thebibliography}
\end{frame}
%%%%%%%%%%%%%%%%%%%%%%%%%%%%%%%%%%%%%%%%%%%%%%%%%%%%%%55
\begin{frame}
\begin{center}

{\color{blue}{\Huge{\bf СПАСИБО ЗА ВНИМАНИЕ!!!}}}
\end{center}
\end{frame}



\end{document}

